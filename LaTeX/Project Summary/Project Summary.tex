% do not change these two lines (this is a hard requirement
% there is one exception: you might replace oneside by twoside in case you deliver 
% the printed version in the accordant format
\documentclass[11pt,titlepage,oneside,openany]{book}
\usepackage{times}


\usepackage{graphicx}
\usepackage{latexsym}
\usepackage{amsmath}
\usepackage{amssymb}

\usepackage{ntheorem}

% \usepackage{paralist}
\usepackage{tabularx}

% this packaes are useful for nice algorithms
\usepackage{algorithm}
\usepackage{algorithmic}

%this package adds " " after commands, allows us a nicer
%formatting of texts
\usepackage{xspace}
\usepackage[colorlinks = false, pdfborder={0 0 0}]{hyperref}

% well, when your work is concerned with definitions, proposition and so on, we suggest this
% feel free to add Corrolary, Theorem or whatever you need
\newtheorem{definition}{Definition}
\newtheorem{proposition}{Proposition}


% its always useful to have some shortcuts (some are specific for algorithms
% if you do not like your formating you can change it here (instead of scanning through the whole text)
\renewcommand{\algorithmiccomment}[1]{\ensuremath{\rhd} \textit{#1}}
\def\MYCALL#1#2{{\small\textsc{#1}}(\textup{#2})}
\def\MYSET#1{\scshape{#1}}
\def\MYAND{\textbf{ and }}
\def\MYOR{\textbf{ or }}
\def\MYNOT{\textbf{ not }}
\def\MYTHROW{\textbf{ throw }}
\def\MYBREAK{\textbf{break }}
\def\MYEXCEPT#1{\scshape{#1}}
\def\MYTO{\textbf{ to }}
\def\MYNIL{\textsc{Nil}}
\def\MYUNKNOWN{ unknown }
% simple stuff (not all of this is used in this examples thesis
\def\INT{{\mathcal I}} % interpretation
\def\ONT{{\mathcal O}} % ontology
\def\SEM{{\mathcal S}} % alignment semantic
\def\ALI{{\mathcal A}} % alignment
\def\USE{{\mathcal U}} % set of unsatisfiable entities
\def\CON{{\mathcal C}} % conflict set
\def\DIA{\Delta} % diagnosis
% mups and mips
\def\MUP{{\mathcal M}} % ontology
\def\MIP{{\mathcal M}} % ontology
% distributed and local entities
\newcommand{\cc}[2]{\mathit{#1}\hspace{-1pt} \# \hspace{-1pt} \mathit{#2}}
\newcommand{\cx}[1]{\mathit{#1}}
% complex stuff
\def\MER#1#2#3#4{#1 \cup_{#3}^{#2} #4} % merged ontology
\def\MUPALL#1#2#3#4#5{\textit{MUPS}_{#1}\left(#2, #3, #4, #5\right)} % the set of all mups for some concept
\def\MIPALL#1#2{\textit{MIPS}_{#1}\left(#2\right)} % the set of all mips


\begin{document}

\pagenumbering{roman}
% lets go for the title page, something like this should be okay
\begin{titlepage}
	\vspace*{2cm}
  \begin{center}
   {\Large Extraction of Goals from Premier League Match Reports using Natural Language Processing Techniques\\}
   \vspace{2cm} 
   {Student  Project\\}
   \vspace{2cm}
   {presented by\\
    Jochen H\"{u}l\ss \xspace  \&  Matthias Rabus \\
    Matriculation Number 1376749 \& 1207834 \\
   }
   \vspace{1cm} 
   {submitted to the\\
    Chair of Information Systems V\\
    Prof.\ Dr.\ Christian\ Bizer\\
    University Mannheim\\} \vspace{2cm}
   {Mai 2013}
  \end{center}
\end{titlepage} 

% no lets make some add some table of contents
\tableofcontents
\newpage

%\listofalgorithms

%\listoffigures

%\listoftables

% evntuelly you might add something like this
% \listtheorems{definition}
% \listtheorems{proposition}

\newpage


% okay, start new numbering ... here is where it really starts
\pagenumbering{arabic}

\chapter{Project Summary}
\section{Application Domain and Goals}
Our interest in looking into text mining and, particulary, into sport report mining emerges from the overwhelming availability, omnipresence, and ubiquitousness of information on sports in media. 

The project originates from a classical text mining problem. Generally speaking, we would like to deal with the question whether or not Natural Language Processing (NLP) tools are able to extract and correctly classify certain events in a given text. More specifically, our intention is machine-reading written reports on Premier League football matches. The particular event we are looking into is the scoring of a goal. At first sight, this appears rather straightforward, but by taking into account that there are numerous or even infinite ways to express that a team marks a goal, the difficulty to find as many of them as possible is challenging. Thus, a classification of text snippets based on manually found patterns indicating a goal is needed. Furthermore, we would like to enhance our search result by using learning algorithms.  


Domains:

- Information Extraction w/ help of web crawling, NLP, POS Tagging, NER, manually designed patterns matching football goals
- Text Mining: Learning a Classification model on automatically-generated labeled data seeds, bootstrapped negative seeds	

part-of-speech taggers are computer programs for assigning contextually appropriate grammatical descriptors to words in texts;
these approaches are: taggers based on handwritten local rules, taggers based on n-grams automatically derived from text corpora   n-gram are subsets of a text with length n \citep[p.219]{Voutilainen2005} 

Due to the explosion of available textual data, text mining and Information Ex-
traction (IE) from texts have become important topics of study in recent years.
In particular, detection of relations between named entities is a challenging task
to automatically discover new relationships in texts. \citep[p.1]{Cellier2010}

Some previous works use hand-
crafted linguistic IE rules for that task which is time consuming [7,5]. Other
methods based on Machine Learning (ML) techniques [10] give good results but
run as a “black box” \citep[p.1]{Cellier2010}

Sequential pattern mining [1] is a data mining technique that aims at discovering
correlations between events through their order of appearance. Sequential pat-
tern mining is an important field of data mining with broad applications (e.g.,
biology, marketing, security) \citep[p.1]{Cellier2010}

Information extraction (IE) is the automatic identification of selected types of entities, relations, or events in free text. \citep[p.545]{Grishman2005}

The project originates from a classical text mining problem. Generally speaking, we would like to deal with the question whether or not Natural Language Processing (NLP) tools are able to extract and correctly classify certain events in a given text. More specifically, our intention is machine-reading written reports on Premier League football matches. The particular event we are looking into is the scoring of a goal. At first sight, this appears rather straightforward, but by taking into account that there are numerous or even infinite ways to express that a team marks a goal, the difficulty to find as many of them as possible is challenging. Thus, a classification of text snippets based on manually found patterns indicating a goal is needed. Furthermore, we would like to enhance our search result by using learning algorithms.  

While looking for goals in football match reports, we are omitting the detection of team that scored as well as the unique recognition of a goal as many reports refer multiple times to the same goal.

Jochen. Einleitung, Uebertragbarkeit auf andere Text Mining Tasks. Steps to goal

\section{Structure and size of the data set}

Since many NLP tools work best with English language, we would like to increase our chance of success by using football reports of the English Premier League. The BBC sports department offers match reports for every game of the currently ongoing season 2012/2013. The amount of reports is sufficient for the purpose of this project. A Premier League season consists of 38 matchdays with 10 games each. BBC provides one report per game. BBC is a goog datasource because the reports are neutrally written and have no preference for any team. Furthermore, the used language has a higher level and should be, compared to spoken word, easier to analyse. \\
The first step of the project will be collecting the data from the BBC homepage. 	

Matze. siehe outline, punkt 1.2 Struktur input und struktur fuer web mining (satzweise getrennt)

\section{Preprocessing}
Matze: Wie wurde struktur aus vorherigem Abschnitt erreicht? 
Jochen: Gate, Jape-Rules, ngrams

\section{Actual Web Mining}
Matze. Verbindung zur Vorlesung. Web mining techniken. Rapidminer prozess. classifier. Suche nach perfect settings. apply model 

\section{Evaluation}
Jochen. Resultate von Precision und Recall der Classification. Diskussion. Positive und negative Einflüsse auf ergebnisse. Analog fuer ungelabelte Daten 
\section{Conclusion}
Wrap-up.

\bibliographystyle{apalike}
\bibliography{bibliography}
\end{document}
