% do not change these two lines (this is a hard requirement
% there is one exception: you might replace oneside by twoside in case you deliver 
% the printed version in the accordant format
\documentclass[11pt,titlepage,oneside,openany]{book}
\usepackage{times}


\usepackage{graphicx}
\usepackage{latexsym}
\usepackage{amsmath}
\usepackage{amssymb}

\usepackage{ntheorem}

% \usepackage{paralist}
\usepackage{tabularx}

% this packaes are useful for nice algorithms
\usepackage{algorithm}
\usepackage{algorithmic}

%this package adds " " after commands, allows us a nicer
%formatting of texts
\usepackage{xspace}
\usepackage[colorlinks = false, pdfborder={0 0 0}]{hyperref}

% well, when your work is concerned with definitions, proposition and so on, we suggest this
% feel free to add Corrolary, Theorem or whatever you need
\newtheorem{definition}{Definition}
\newtheorem{proposition}{Proposition}


% its always useful to have some shortcuts (some are specific for algorithms
% if you do not like your formating you can change it here (instead of scanning through the whole text)
\renewcommand{\algorithmiccomment}[1]{\ensuremath{\rhd} \textit{#1}}
\def\MYCALL#1#2{{\small\textsc{#1}}(\textup{#2})}
\def\MYSET#1{\scshape{#1}}
\def\MYAND{\textbf{ and }}
\def\MYOR{\textbf{ or }}
\def\MYNOT{\textbf{ not }}
\def\MYTHROW{\textbf{ throw }}
\def\MYBREAK{\textbf{break }}
\def\MYEXCEPT#1{\scshape{#1}}
\def\MYTO{\textbf{ to }}
\def\MYNIL{\textsc{Nil}}
\def\MYUNKNOWN{ unknown }
% simple stuff (not all of this is used in this examples thesis
\def\INT{{\mathcal I}} % interpretation
\def\ONT{{\mathcal O}} % ontology
\def\SEM{{\mathcal S}} % alignment semantic
\def\ALI{{\mathcal A}} % alignment
\def\USE{{\mathcal U}} % set of unsatisfiable entities
\def\CON{{\mathcal C}} % conflict set
\def\DIA{\Delta} % diagnosis
% mups and mips
\def\MUP{{\mathcal M}} % ontology
\def\MIP{{\mathcal M}} % ontology
% distributed and local entities
\newcommand{\cc}[2]{\mathit{#1}\hspace{-1pt} \# \hspace{-1pt} \mathit{#2}}
\newcommand{\cx}[1]{\mathit{#1}}
% complex stuff
\def\MER#1#2#3#4{#1 \cup_{#3}^{#2} #4} % merged ontology
\def\MUPALL#1#2#3#4#5{\textit{MUPS}_{#1}\left(#2, #3, #4, #5\right)} % the set of all mups for some concept
\def\MIPALL#1#2{\textit{MIPS}_{#1}\left(#2\right)} % the set of all mips


\begin{document}

\pagenumbering{roman}
% lets go for the title page, something like this should be okay
\begin{titlepage}
	\vspace*{2cm}
  \begin{center}
   {\Large Extraction of Goals from Premier League Match Reports using Natural Language Processing Techniques\\}
   \vspace{2cm} 
   {Student  Project\\}
   \vspace{2cm}
   {presented by\\
    Jochen H\"{u}l\ss \xspace  \&  Matthias Rabus \\
    Matriculation Number 1376749 \& 1207834 \\
   }
   \vspace{1cm} 
   {submitted to the\\
    Chair of Information Systems V\\
    Prof.\ Dr.\ Christian\ Bizer\\
    University Mannheim\\} \vspace{2cm}
   {Mai 2013}
  \end{center}
\end{titlepage} 

% no lets make some add some table of contents
\tableofcontents
\newpage

%\listofalgorithms

%\listoffigures

%\listoftables

% evntuelly you might add something like this
% \listtheorems{definition}
% \listtheorems{proposition}

\newpage


% okay, start new numbering ... here is where it really starts
\pagenumbering{arabic}

\chapter{Project Summary}
\section{Application domain and goals}
Jochen. Einleitung, Uebertragbarkeit auf andere Text Mining Tasks. Steps to goal

\section{Structure and size of the data set}
Matze. siehe outline, punkt 1.2 Struktur input und struktur fuer web mining (satzweise getrennt)

\section{Preprocessing}
Matze: Wie wurde struktur aus vorherigem Abschnitt erreicht? 
Jochen: Gate

\section{Actual Web Mining}
Matze. Verbindung zur Vorlesung. Web mining techniken. Rapidminer prozess. classifier. Suche nach perfect settings. apply model 

\section{Evaluation}
Jochen. Resultate von Precision und Recall der Classification. Diskussion. Positive und negative Einflüsse auf ergebnisse. Analog fuer ungelabelte Daten 
\section{Conclusion}
Wrap-up.

\bibliographystyle{plain}
\bibliography{bibliography}
\end{document}
